\documentclass[12pt]{article}

\usepackage[sfdefault]{ClearSans}
\usepackage[utf8]{inputenc}
\usepackage[T1]{fontenc}
\usepackage[francais]{babel}
\usepackage{color}
\usepackage[top=2cm, bottom=2cm, left=2cm, right=2cm]{geometry}
\usepackage{eurosym}
\usepackage{graphicx}

\pagestyle{plain}
\title{Micro-projet}
\author{Hubert Hirtz et Camille Schnell}
\date{22 octobre 2018}
\begin{document}
\maketitle
\renewcommand{\contentsname}{Sommaire}
\tableofcontents
\newpage
\section{Introduction}
  L'objectif de ce micro-projet est d'implémenter un système d'e-shopping. Afin de faire cela, nous gérons des utilisateurs, ainsi que des articles, en base de données. Nous avons pour cela défini l'architecture du projet, en utilisant des design patterns répondant aux problèmes posés. \\
  Nous allons dans un premier présenter nos choix architecturaux, et indiquer quels designs patterns nous avons mis en place. Nous expliquerons ensuite la solution globale du projet, ainsi que la gestion des données. Nous fournirons enfin un manuel utilisateur de notre application, et nous finirons par les difficultés que nous avons pu rencontrer lors du déroulement du micro-projet.
\newpage
\section{Architecture et design patterns}
\newpage
\section{Solution et gestion des données}
\newpage
\section{Manuel d'utilisation}
    \subsection{Lancement du script de démonstration}
    Pour lancer la démonstration, ...
    \subsection{Scénarios clients prédéfinis}
    Nous avons implémenté deux scénarios prédéfinis afin de montrer le fonctionnement de notre application.
    \subsubsection{Premier scénario}
    Ce premier scénario permet de ...
    \subsubsection{Deuxième scénario}
    Ce deuxième scénario a pour objectif de tester ...
\newpage
\section{Difficultés rencontrées}

\newpage
\section{Conclusion}

\end{document}
